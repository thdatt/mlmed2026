\documentclass[conference]{IEEEtran}

\usepackage{graphicx}
\usepackage{caption}
\usepackage{amsmath}
\usepackage{booktabs}
\usepackage{float}
\raggedbottom
\setlength{\textfloatsep}{8pt}
\setlength{\floatsep}{6pt}
\setlength{\intextsep}{6pt}
\title{Practical Work 2 -- Measurement of Fetal Head Circumference}

\author{Ngo Thanh Dat -- 23BI14090}

\begin{document}

\maketitle

\section{Introduction}
Head circumference (HC) is one of the basic biometric parameters used to assess fetal size.
Together with biparietal diameter (BPD), abdominal circumference (AC), and femur length (FL),
HC is commonly used to estimate fetal weight.
In this practical work, a machine learning approach is applied to estimate fetal head circumference
from ultrasound images using pixel-based regression.

\section{Dataset Overview}

The dataset contains fetal ultrasound images stored in a training set folder, along with a CSV file providing head circumference (HC) measurements.
The test set does not include HC values, therefore the training data is split into training and validation subsets for evaluation.

\begin{itemize}
    \item Training set: ultrasound images with corresponding HC labels
    \item CSV file: image names and head circumference values (mm)
\end{itemize}


\section{Exploratory Data Analysis}

Before training the regression model, the dataset is briefly explored to understand the head circumference values and the quality of the ultrasound images.

Figure~\ref{fig:hc_dist} shows the distribution of head circumference values in the training set.
The values span a wide range and are continuous, indicating that the problem should be solved using regression rather than classification.
Most samples are concentrated around the middle range, which should be considered when interpreting model performance.


\begin{figure}[H]
    \centering
    \includegraphics[width=0.8\linewidth]{HeadCircum.png}
    \caption{Distribution of head circumference values in the training set.}
    \label{fig:hc_dist}
\end{figure}
The ultrasound images vary in quality due to noise and differences in acquisition conditions.
To examine this variation, the standard deviation of pixel intensities is computed for a subset of images.
As shown in Figure~\ref{fig:pixel_std}, the contrast of the images is not consistent across samples, which makes the prediction task challenging.


\begin{figure}[H]
    \centering
    \includegraphics[width=0.8\linewidth]{PixelIntensity.png}
    \caption{Distribution of pixel intensity standard deviation across training images.}
    \label{fig:pixel_std}
\end{figure}

Figure~\ref{fig:sample_images} presents several randomly selected ultrasound images with their corresponding head circumference values.
The fetal head boundaries are often unclear and affected by noise.
These examples illustrate why a simple pixel-based approach may not capture precise anatomical details, but can still provide useful information for approximate head circumference estimation.

\begin{figure}[H]
    \centering
    \includegraphics[width=\linewidth]{Testing.png}
    \caption{Sample ultrasound images with corresponding head circumference values (mm).}
    \label{fig:sample_images}
\end{figure}
\section{Methodology: Pixel-based Regression}

In this work, fetal head circumference estimation is treated as a regression problem.
A simple machine learning approach is used, where ultrasound images are represented directly by their pixel values.
This design keeps the method easy to understand and avoids complex techniques such as segmentation or deep learning.

\subsection{Image Preprocessing}

All ultrasound images are converted to grayscale and resized to a fixed resolution of $64 \times 64$ pixels.
Pixel values are normalized to the range $[0,1]$ to reduce differences caused by imaging conditions.

\subsection{Feature Representation}

After preprocessing, each image is flattened into a one-dimensional vector of 4096 pixel values.
Each value represents the intensity of one pixel in the ultrasound image.
These pixel vectors are used directly as input features for the regression model.

\subsection{Regression Model and Evaluation}

A Random Forest Regressor is used to predict head circumference from the pixel-based features.
The dataset is split into training and validation sets using an 80/20 ratio.
Model performance is evaluated using Mean Absolute Error (MAE), which measures the average prediction error in millimeters.

\subsection{Hyperparameter Experiment}

Different Random Forest configurations are tested by varying the number of trees and the maximum tree depth.
This experiment helps evaluate the stability of the model under different parameter settings.

\section{Results}

Table~\ref{tab:rf_comparison} presents the performance of the Random Forest regressor under different hyperparameter configurations.
The comparison is based on MAE values obtained from the validation set.

\begin{table}[H]
    \centering
    \caption{Random Forest Performance under Different Hyperparameters}
    \label{tab:rf_comparison}
    \begin{tabular}{ccc}
        \toprule
        \textbf{n\_estimators} & \textbf{max\_depth} & \textbf{MAE (mm)} \\
        \midrule
        50  & None & 29.05 \\
        50  & 10   & 29.59 \\
        50  & 20   & 28.95 \\
        100 & None & 28.94 \\
        100 & 10   & 29.12 \\
        100 & 20   & 28.95 \\
        200 & None & \textbf{28.75} \\
        200 & 10   & 28.87 \\
        200 & 20   & 28.89 \\
        \bottomrule
    \end{tabular}
\end{table}
The results indicate that increasing the number of trees generally leads to a small improvement in prediction accuracy.
The best performance is obtained with 200 trees and unlimited tree depth, achieving a minimum MAE of 28.75 mm.

Limiting the maximum tree depth does not consistently improve performance and can slightly reduce accuracy.
Overall, the Random Forest model shows stable behavior across different hyperparameter settings, suggesting that the approach is robust to moderate changes in model complexity.

\section{Conclusion}

This practical work presents a simple machine learning approach for estimating fetal head circumference from ultrasound images.
By using pixel-based image features and a Random Forest regression model, reasonable prediction performance is achieved without relying on complex techniques such as segmentation or deep learning.
The results demonstrate that a straightforward and interpretable method can serve as an effective baseline for fetal head circumference estimation.


\end{document}
