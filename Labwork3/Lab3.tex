\documentclass[conference]{IEEEtran}

\usepackage{graphicx}
\usepackage{float}
\raggedbottom

\title{Practical Work 3 -- COVID-19 Infection Segmentation from Chest X-ray Images}
\author{Ngo Thanh Dat -- 23BI14090}

\begin{document}
\maketitle

% =========================
\section{Introduction}

COVID-19 infection mainly affects lung tissues and can be observed in chest X-ray images.
Automatically segmenting infection regions from X-ray images can support medical analysis.

In this practical work, a deep learning approach is used to segment COVID-19 infection regions.
Because infection appears only inside the lungs, lung masks are first applied to restrict the region of interest.
A single segmentation model is then trained to detect infection regions inside the lung area.

% =========================
\section{Dataset Overview}

The COVID-QU-Ex dataset is used in this experiment.
The dataset provides chest X-ray images together with lung segmentation masks and infection segmentation masks.

The data is divided into training, validation, and test subsets.
In this work, only images from the COVID-19 category are used for infection segmentation.

% =========================
\section{Preprocessing}

Before training the segmentation model, lung masks are applied to the X-ray images.
Each X-ray image is multiplied by its corresponding lung mask so that only lung regions are preserved.

This preprocessing step removes irrelevant background regions and helps the model focus on areas where infection may appear.

% =========================
\section{Methodology}

\subsection{Dataset Preparation}

Each training sample consists of:
\begin{itemize}
    \item Input: masked chest X-ray image containing only lung regions
    \item Target: infection segmentation mask
\end{itemize}

All images are converted to grayscale and normalized to the range $[0,1]$.
Lung masks are used only as a preprocessing step and are not treated as learning targets.

\subsection{Segmentation Model}

A simplified U-Net architecture is used for infection segmentation.
The model follows an encoder-decoder structure with convolutional layers and ReLU activations.
The output of the model is a single-channel infection probability map.

Only one segmentation model is trained, as required by the assignment.

\subsection{Loss Function and Training Strategy}

To address the imbalance between infection and background pixels, a combined loss function is used.
The loss is defined as the average of Binary Cross Entropy loss and Dice loss.

Binary Cross Entropy loss evaluates pixel-wise classification accuracy, while Dice loss encourages better overlap between predicted and ground-truth infection regions.
This combination improves learning stability for small and sparse infection areas.

The model is trained using the Adam optimizer with a fixed learning rate.
Training is performed for a small number of epochs to demonstrate correct learning behavior.

% =========================
\section{Results}

During training, the loss decreases steadily across epochs, indicating successful learning.
Figure~\ref{fig:result} shows an example of segmentation results.

\begin{figure}[H]
    \centering
    \includegraphics[width=\linewidth]{Prediction.png}
    \caption{Example segmentation result. From left to right: masked X-ray image, ground-truth infection mask, and predicted infection mask.}
    \label{fig:result}
\end{figure}

The predicted infection regions are mainly located inside the lung area and show reasonable agreement with the ground truth.

% =========================
\section{Conclusion}

This work presents a simple and effective pipeline for COVID-19 infection segmentation from chest X-ray images.
By using lung masks as a preprocessing step, the segmentation model focuses on relevant anatomical regions and avoids background noise.

The use of a combined Binary Cross Entropy and Dice loss helps handle class imbalance and improves region-level segmentation quality.
Overall, the proposed approach provides a clear and easy-to-understand baseline for medical image segmentation tasks.

\end{document}
